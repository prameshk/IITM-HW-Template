\documentclass{article}
\usepackage{graphicx,fancyhdr,amsmath,amssymb,amsthm,subfig,url,hyperref}
\usepackage[margin=1in]{geometry}

%----------------------- Macros and Definitions --------------------------

%%% FILL THIS OUT
\newcommand{\studentname}{Pramesh Kumar}
\newcommand{\suid}{prameshk}
\newcommand{\exerciseset}{Homework 1}
%%% END



\renewcommand{\theenumi}{\bf \Alph{enumi}}

%\theoremstyle{plain}
%\newtheorem{theorem}{Theorem}
%\newtheorem{lemma}[theorem]{Lemma}

\fancypagestyle{plain}{}
\pagestyle{fancy}
\fancyhf{}
\fancyhead[RO,LE]{\sffamily\bfseries\large Indian Institute of Technology Madras}
\fancyhead[LO,RE]{\sffamily\bfseries\large CE 3205: Traffic Engineering}
\fancyfoot[LO,RE]{\sffamily\bfseries\large \studentname: \suid @iitm.ac.in}
\fancyfoot[RO,LE]{\sffamily\bfseries\thepage}
\renewcommand{\headrulewidth}{1pt}
\renewcommand{\footrulewidth}{1pt}

\graphicspath{{figures/}}

%-------------------------------- Title ----------------------------------

\title{CE 3205 \exerciseset}
\author{\studentname \qquad IITM-ID: \suid}

%--------------------------------- Text ----------------------------------

\begin{document}
\maketitle

\section*{Problem 1}
\begin{enumerate}
\item %A
One can write the set of feasible link flows as below:
\begin{eqnarray}
	\left\{x \in \mathbb{R}^{\vert A \vert} : \sum_{j \in FS(i)} x_{ij}^{rs}  - \sum_{j \in BS(i)}  x_{ji}^{rs} = b_{i}^{rs}, \forall rs \in Z^2\right\}, 
\end{eqnarray}
	where, 

\item %B
Next one here.

\item %C
42

\end{enumerate}

\section*{Problem 2}
\begin{enumerate}
\item %A

\item %B

\item %C

\item %D

\item %E

\end{enumerate}


\end{document}
